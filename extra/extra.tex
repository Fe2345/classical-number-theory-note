\ifx\allfiles\undefined
\documentclass[12pt, a4paper, oneside, UTF8]{ctexbook}
\setCJKmainfont{SimSun}
\def\path{../config}
\input{../config/_config}
\begin{document}
	% \input{../config/cover}
	\else
	\fi
	%标题
	\chapter{附录}
	这一部分中,对于正文中因为逻辑结构无法提及的部分,进行补充。包括特殊函数,有趣的数学概念,一些命题的全新解法,以及难以推导的公式
	证明可能使用复分析、实分析、泛函等超纲内容
	%--------------------正文---------------------------
	
	%附录:不定积分初等性判定
	\section{原函数初等性的判定方法}
		\subsection{切比雪夫定理}
			\begin{them}{切比雪夫定理}{}
				设$m,n,p\in \Q-\{0\}$,那么以下积分
				\begin{equation}
					\int x^m(a+bx^n)^p \d x
				\end{equation}
				初等的充要条件是:$p,\frac{m+1}{n},\frac{m+1}{n}+p$中至少有一个为整数
			\end{them}
		\subsection{刘维尔定理}
			在介绍刘维尔定理前,需要先介绍一些微分代数的概念:
			
			首先我们扩展微分的概念。我们将满足类似乘法、除法微分性质的泛函也称为微分。
			
			先引入微分域及其常数域
			\begin{para}{0}
				\point{微分域}
					\begin{defn}{微分域}{}
						一个由函数组成的域$F$及其上的一个算子$\delta:F \rightarrow F$,如果$\forall f,g \in F$有:
						
						$\ding{172} \delta(f+g) = \delta(f)+\delta(g)$
						
						$\ding{173} \delta(fg) = \delta(f)\cdot g+f \cdot \delta(g)$
						
						那么称$(F,\delta)$是一个微分域
					\end{defn}
					容易验证$\delta$是线性算子,于是我们有时简记$\delta(f)$为$\delta f$
					\begin{defn}{微分域的常数域}{}
						微分域$(F,\delta)$的常数域定义为:
						
						$Con (F,\delta) =\{f \in F| \delta f = 0\}$
					\end{defn}
					同时定义域的扩张:
					\begin{defn}{域的扩张}{}
						设$F,K$是两个域,并且$K$是满足$F \subseteq K$且包含$h \subseteq K$的最小域(即$K$是任何满足上述条件的域的子域),记作$K = F(h)$
					\end{defn}
					作为接下来内容的预备,我们先验证那些显然的微分性质:
					\begin{proposition}
						$\delta C = 0$,其中$C$为常数 
					\end{proposition}
					\begin{proof}
						只需要验证$\delta 1 = 0$
						
						那么有:$\delta (1\cdot 1) = \delta 1 \cdot 1 + 1 \cdot \delta 1=2\delta 1$
						
						于是有$\delta 1 = 0$,利用微分的线性即得证。
					\end{proof}
					\begin{proposition}
						$\delta \left(\frac{f}{g}\right)=\frac{\delta f \cdot g - f \delta g}{g^2}$
					\end{proposition}
					\begin{proof}
						首先推导$\delta \left(\frac{1}{g}\right)$
						
						$\because \delta 1 = \delta \left(g \cdot \frac{1}{g}\right) = 0$
						
						$\Rightarrow \delta g \frac{1}{g}+g \delta \left(\frac{1}{g}\right)=0$
						
						$\Rightarrow \delta \left(\frac{1}{g}\right) = -\frac{\delta g}{g^2}$
						
						于是$\delta \left(\frac{f}{g}\right) = \delta \left(f \cdot \frac{1}{g}\right)$
						
						$=\delta f \frac{1}{g}-f \frac{\delta g}{g^2} = \frac{\delta f \cdot g - f \delta g}{g^2}$
					\end{proof}
				\point{微分域的初等扩张}
					接下来讨论什么是“初等”的函数。
					\begin{defn}{微分域的初等扩张}{}
						设$(F,\delta),(K,\delta)$是两个微分域,$h \in K$并且$K = F(h)$,那么:
						
						$\ding{172}$如果存在$F$中的一个多项式$p(x) \in F[x]$,有$p(h)=0$,那么称$h$是$F$的一个代数元素,$K=F(h)$是$F$的单代数扩张
						
						$\ding{173}$如果存在$F$中的一个函数$f$,使得$\delta h = \frac{\delta f}{f}$,那么称$K=F(h)$是$F$的单对数扩张
						
						$\ding{173}$如果存在$F$中的一个函数$f$,使得$\frac{\delta h}{h} = \delta f$,那么称$K=F(h)$是$F$的单指数扩张。
						
						单对数扩张和单指数扩张统称为单超越扩张,其对应的$h$称为$F$的超越元素;以上三种扩张统称为单初等扩张
						
						有限次初等扩张的复合称为初等扩张
					\end{defn}
					我们也可以在此以另外的方式定义出初等函数:
					\begin{defn}{初等函数}{}
						如果函数$f$处于微分域$\left(C(x),\frac{\d}{\d x}\right)$的某个初等扩张中,那么称$f$是一个初等函数
					\end{defn}
					接下来就可以给出刘维尔定理了。
				\point{刘维尔定理}
					\begin{them}{刘维尔定理}
						设$(F,\delta),(K,\delta)$是两个微分域,$K$是$F$的初等扩张,并且$Con(F,\delta)=Con(K,\delta)$,且$\forall f \in F,\exists g \in K,s.t.\delta g = f$
						
						那么一定$\exists c_1,\cdots,c_n \in Con(F,\delta),u_1,\cdots,u_n,v \in F$,使得
						\begin{equation}
							g = \sum\limits_{i=1}^{n} c_i \ln (u_i)+v
						\end{equation}
					\end{them}
			\end{para}
			
			
	%附录:超越积分的特殊解法
	\section{一些超越积分的特殊解法}
		\subsection{Direchlet积分}
		
	
	\ifx\allfiles\undefined
\end{document}
\fi