\ifx\allfiles\undefined
\documentclass[12pt, a4paper, oneside, UTF8]{ctexbook}
\def\path{../config}
\input{../config/_config}
\begin{document}
% \input{../config/cover}
\else
\fi
%标题
\chapter{唯一分解}
	\section{$\Z$上的唯一分解}
		\subsection{整除和素数}
			作为数论的基础,我们需要先研究素数。
			\begin{defn}{整除}{}
				设$a,b \in \Z,\exists c \in \Z,\text{s.t. }ac=b$,那么我们称$a$整除$b$,记作$a \mid b$;
				
				否则,我们称$a$不整除$b$,记作$a \nmid b$
			\end{defn}
			有了整除的定义,我们就可以定义素数的概念:
			\begin{defn}{素数}{}
				设$p \in \N,p \geqslant 2$,如果$a \mid p,a \in \N \Leftrightarrow a = 1 \vee a = p$
				
				那么我们称$p$是一个素数,否则称其是个合数。
			\end{defn}
			我们研究素数的第一个目的是一个比较显然的事实:所有大于1的整数都能写成若干素数的乘积。我们还将看到,这个分解是唯一的。
			
			我们先引入以下概念。
			\begin{defn}{素数的指数}{}
				设有素数$p \in \N$和$n \in \Z$
				
				如果$p \mid n$,那么我们记满足$p^k \mid n$的最大正整数$k$为$ord_p n$,称为$p$的指数
				
				特别地,如果$p \nmid n$,我们定义$ord_p n = 0$;如果$n =0$,我们定义$ord_p n = +\infty$
			\end{defn}
		\subsection{算数基本定理}
			我们先给出一个基本引理,它证明了可分解但是没有证明唯一性。
			\begin{lemma}{大于1的正整数的可分解性}{}
				每一个大于$1$的正整数都可以写成有限个素数的乘积。
			\end{lemma}
			\begin{proof}
				假设命题不成立。于是一定有一个$N$为不满足此性质的最小正整数。
				
				首先,$N$不可能是素数,否则它就可以写成自身,与假设矛盾。
				
				于是,一定$\exists m,n,1<m < N,1<n<N$,满足$mn=N$
				
				那么,$m,n$也不能满足命题所说的性质,否则$N$也就有了这一性质,与假设矛盾。
				
				但是,这样$N$就不再是最小的满足此性质的正整数了。与假设矛盾,命题得证
			\end{proof}
			接下来我们证明唯一性。为了证明这一部分,我们将提出一系列全新概念,包括互素、最大公因数等。
			\begin{defn}{$\Z$上生成的子模}{}
				我们定义:
				
				$\Z \text{-span} (a_1,\cdots,a_n) := \{a_1\cdot q_1 +\cdots+a_n\cdot q_n | q_i \in \Z\}$
				
				称为$\Z$上由$a_1,\cdots,a_n$生成的子模
			\end{defn}
			我们先证明一个引理:由两个数在整数集上生成的子模,其实也可以只由一个数生成。
			\begin{lemma}{}{}
				对于$\Z\text{-span}(a,b),\exists c \in \Z,\text{s.t. }\Z\text{-span}(a,b)=\Z\text{-span}(c)$
			\end{lemma}
			\begin{proof}
				如果$a=b=0$,那么命题是平凡的。
				
				如果$a,b$中至少有一者不为$0$,那么一定能找到$\Z\text{-span}(a,b)$的最小正元素$c$。
				
				由定义易知,$\Z\text{-span}(c) \subseteq \Z\text{-span}(a,b)$
				
				现在取$\forall d \in \Z\text{-span}(a,b)$,那么一定$\exists q,r \in \Z,0 \leqslant r \leqslant c,d=qc+r$
				
				那么$r=d-qc$,因为$c,d\in \Z\text{-span}(a,b)$,于是$r \in \Z\text{-span}(a,b)$
				
				于是一定有$r=0$因为我们已经假设$c$是$\Z\text{-span}(a,b)$最小的正整数,而$0 \leqslant r$。
				
				那么$d=qc \in \Z\text{-span}(a,b)$,于是$\Z\text{-span}(a,b) \subseteq \Z\text{-span}(c)$,于是命题得证。
			\end{proof}
			这个数实际上就是两个数的最大公因数,我们先给出定义,随后给出证明
			\begin{defn}{公因数}
				对于$a,b \in \Z$,如果$c\in\Z$满足$c\mid a,c\mid b$,那么称$c$是$a,b$的最大公因数。
				
				如果又有:对于任何$a,b$的公因数$d,d \mid c$,那么称$c$是$a,b$的最大公因数,记作$(a,b)$
			\end{defn}
			\begin{lemma}{}{}
				如果$\Z\text{-span}(a,b)=\Z\text{-span}(c)$,那么$c$是$a,b$的最大公因数
			\end{lemma}
			\begin{proof}
				取$a,b$的一个公因数$d\mid a,d\mid b$,由$\Z\text{-span}(a,b)$的定义可知,$\forall e \in \Z\text{-span}(a,b),d \mid e$
				
				那么$\forall e \in \Z\text{-span}(c),d \mid e$,那么$d \mid c$,命题得证。
			\end{proof}
			最大公因数是$1$的情况比较特别,我们称之为互素
			\begin{defn}{互素}{}
				如果$a,b$的公因数仅有$1,-1$,那么我们称$a,b$互素,记作$(a,b)=1$
			\end{defn}
			互素有以下显然的性质:
			\begin{lemma}{}{}
				$(a,b)=1 \Leftrightarrow \exists r,q \in \Z,\text{s.t. }ra+qb=1$
			\end{lemma}
			\begin{proof}
				只需注意到$(a,b)=1 \Leftrightarrow \Z\text{-span}(a,b)=\Z\text{-span}(1)$,于是命题得证。
			\end{proof}
			\begin{lemma}{}{}
				若$a \mid bc,(a,b)=1$,那么$a\mid c$
			\end{lemma}
			\begin{proof}
				因为$(a,b)=1$,所以$\exists r,q\in \Z,\text{s,t. }ra+qb=1$
				
				所以$rac+qbc=c$,但是我们知道$a\mid rac,a\mid qbc$,于是$a\mid c$
			\end{proof}
			\begin{lemma}{}{}
				如果$p$是一个素数,$p \mid bc$,那么$p \mid b,p\mid c$至少有一者成立。
			\end{lemma}
			\begin{proof}
				因为$p$是一个素数,所以$(p,b)$或者为$1$,或者为$p$。
				
				如果$(p,b)=p$那么一定有$p \mid b$,命题成立;
				
				如果$(p,b)=1$,那么$p,b$的公因数仅有$1,-1$,那么必定有$p \mid c$
			\end{proof}
			最后,为了指出唯一分解中素数的指数,我们给出最后一个引理。
			\begin{lemma}{}{}
				如果$p$为素数,$a,b\in \Z$,那么$ord_p ab=ord_p a + ord_p b$
			\end{lemma}
			\begin{proof}
				首先处理一些特殊情形:
				
				如果$a,b$中至少一者为$0$,不妨假设$a=0$,那么有$+\infty = +\infty + ord_p b$,因为$ord_p b$或者有限,或者为正无穷,于是命题成立;
				
				如果$p \nmid ab,p\nmid a,p\nmid b$,那么有$0=0+0$,命题成立;
				
				如果$p \mid ab$,但是$p \mid a$和$p \mid b$仅一者成立,那么不妨假设$p \nmid a$,于是依上面的引理,有$p \mid b$
				
				假设$ord_p b = n$,于是$\exists c,b=c\cdot p^n,p \nmid c$
				
				代入得:$ab=p^n\cdot (ac)$,但是$p \nmid a,p\nmid c$,那么$p \nmid ac$,即$ord_p ab=n$,于是$ord_p ab = 0+ ord_p b$,命题成立。
				
				最后假设$p \mid ab,p \mid a,p \mid b$,设$ord_p a = m,ord_p b = n$,于是$\exists c,d,a = c\cdot p^m,b=d\cdot p^n,p \nmid c,p\nmid d$
				
				那么$ab=(cd)\cdot p^{m+n}$,但是$p \nmid cd$,于是有$ord_p ab=ord_p a + ord_p b$,命题成立。
			\end{proof}
			至此,我们可以开始证明算数基本定理了。
			\begin{them}{算数基本定理}{}
				$\forall n \in \Z,n \neq 0$
				
				\begin{equation}
					n = (\text{sgn } n) \prod_{p} p^{ord_p |n|}
				\end{equation}
				其中求积下标$p$对全体素数求积
			\end{them}
			\begin{proof}
				先对$n > 1$考虑,此时有$n=\prod_{p} p^{a_p}$,其中$a_p$是未知的。
				
				取任意一个素数$q$,那么有$ord_q n=\sum\limits_{p} a_p\cdot ord_q (p)$
				
				只需注意到如果$q \neq p,ord_q (p)=0$,即得$a_q = ord_p (q)$,于是此情形下命题成立。
				
				对于$n=1$的情形,$ord_p (1)=0$恒成立,于是命题也成立。
				
				$n <0$的情形,在已经证明了上述事实后是显然的。 
			\end{proof}
	\section{$K[x]$上的唯一分解}
	\section{主理想整环上的唯一分解}
		\subsection{主理想整环}
			\begin{defn}{主理想整环}{}
				设$R$是一个交换环,如果它的任何一个理想$I \subseteq R$都是主理想,即
				
				$\exists c,I = (c) = \{rc | c \in R\}$
				
				那么我们称$R$是一个主理想整环
			\end{defn}
			在初等数论中,我们更多考虑的所谓欧几里得环
			\begin{defn}{欧几里得环}{}
				设$R$是一个交换环,如果$\exists \lambda : R-{0_R} \rightarrow \N$,称为欧几里得函数,满足
				
				$\forall a,b \in R,\exists c,d \in R,a = cb+d$,$d$满足$d=0$或者$\lambda (d) < \lambda (b)$
				
				那么称$R$是一个交换环
			\end{defn}
			本节中我们把主要精力放在主理想整环上,我们接下来定义主理想整环上的一系列概念
			\begin{defn}{整除}{}
				设$R$是一个主理想整环,$a,b\in R$,如果$\exists c \in R,ac = b$,那么称$a$整除$b$,记作$a \mid b$
				
				否则,称$a$不整除$b$,记作$a \nmid b$
			\end{defn}
			这个定义也可以写成一个等价形式:$(b) \subseteq (a)$
			\begin{defn}{不可约元}{}
				主理想整环$R$中的元素$a$如果满足:
				
				如果$b \mid a$,那么或者$b$可逆,或者$a$与$b$相伴,那么我们称$a$是$R$中的不可约元
			\end{defn}
			\begin{defn}{素元}{}
				主理想整环$R$中的元素$p$如果满足
				
				$p \mid ab \Rightarrow p \mid a \vee n \mid b$,
				
				那么称$p$是$R$上的素元
			\end{defn}
			\begin{defn}{公因子}{}
				主理想整环$R$中的元素$a,b$,如果$c \in R$满足$c \mid a,c\mid b$,那么称$c$是$a,b$的公因子;
				
				如果还满足:任意$a,b$的公因子$d,d\mid c$,那么称$c$是$a,b$的最大公因子,记作$(a,b)=c$
			\end{defn}
			\begin{defn}{互素}{}
				如果$(a,b)=c \Leftrightarrow c \text{可逆}$,那么称$a,b$互素。
			\end{defn}
		\subsection{主理想整环的唯一分解定理}
			\begin{proposition}
				任何一个欧几里得环都是主理想子环
			\end{proposition}
			\begin{proof}
				设$R$是一个欧几里得环,取$R$上的任意一个理想$I \subseteq R$
				
				考虑集合$S = \{\lambda (k)|k \in I,k \neq 0\}$,取其中的元素$a$满足$\forall b \in S,\lambda (a) \leqslant \lambda (b)$。
				
				因为$a \in I$,按照理想的定义一定有$(a) \subseteq I$,我们接下来证明相反方向的式子。
				
				再取$b \in I$,那么$\exists c,d \in I,b=ca+d$
				
				现在假设$d=0$,那么有$b=ca$,$I \subseteq (a)$自然成立
				
				如果$d \neq 0$,那么就有$\lambda (d) <\lambda (a)$,但是我们本来就假设$\lambda (a)$应该是所有$\lambda (k) ,k\neq 0$中最小的,矛盾。
				
				于是命题得证
			\end{proof}
			\begin{proposition}
				主理想整环中的任何两个元素$a,b \in R$,
				
				$c = (a,b)$都存在,并且有$R\text{-span}(a,b) = (c)$
			\end{proposition}
			\begin{proof}
				显然,$R\text{-span}(a,b)$也是$R$的一个理想,于是$\exists c \in R,\text{s.t. } R\text{-span}(a,b)= (c)$
				
				那么有$(a) \subseteq (c),(b)\subseteq (c)$,于是$c\mid a,c\mid b$,即$c$是$a,b$的公因数
				
				现在假设$d \mid a,d\mid b$,那么$(a) \subseteq (d),(b)\subseteq (d)$
				
				于是一定有$R\text{-span}(a,b) \subseteq (d)$,即$(c) \subseteq (d)$,那么$d\mid c$,于是命题得证。
			\end{proof}
			\begin{corollary}{}{}
				如果$a,b$互素,那么$R\text{-span}(a,b)=R$
			\end{corollary}
			\begin{proof}
				如果$a,b$互素,那么$(a,b)=1$,但是$(1)=R$,于是命题得证。
			\end{proof}
			\begin{corollary}{}{}
				不可约元$p$同时也是素元
			\end{corollary}
			\begin{proof}
				只需证明$p \mid ab,p \nmid a \Rightarrow p \mid b$
				
				因为$p$不可约,所以$p$仅有可逆元以及与$p$相伴的元素为因子
				
				但是$p \nmid a$,那么一定有$R\text{-span}(p,a)=1$,于是$R\text{-span}(pb,ab)=(b)$
				
				又因为$p \mid pb,p\mid ab$,所以$(pb)\subseteq (p),(ab)\subseteq (p)$
				
				$\Rightarrow R\text{-span}(pb,ab) \subseteq (p) \Rightarrow (b) \subseteq (p) \Rightarrow p \mid b$,于是命题得证
			\end{proof}
			\begin{lemma}{}{}
				取一个主理想链$(a_1)\subseteq (a_2) \subseteq \cdots \subseteq (a_i) \subseteq \cdots$
				
				那么,一定$\exists k,\forall i \in \N,(a_k)=(a_{k+i})$,也就是说这个链在某一元素开始到达极限
			\end{lemma}
			\begin{proof}
				设$I = \bigcup_{i=1}^{\infty} (a_i)$,那么$\exists a ,I=(a)$
				
				于是一定能找到一个$k,a \in (a_k)$,于是一定有$(a) \subseteq (a_k) \Rightarrow I \subseteq (a_k)$,那么命题得证
			\end{proof}
			\begin{proposition}
				主理想整环$R$中的每一个非零不可逆元素都可以写成一系列不可约元的乘积
			\end{proposition}
			\begin{proof}
				取$a \in R$,$a$非零且不可逆。我们首先证明$a$可被不可约元整除。
				
				假设$a$不可约,那证明完毕;如果可约,那么有$a = a_1 b_1$,我们如此拆分$a_i = a_{i+1}b_{i+1}$,如果有一步中$a_{n}$不可约,那么证明完毕
				
				如果找不到这样的一不,那么按照整除的定义,有$(a) \subseteq (a_1) \subseteq \cdots \subseteq (a_k) \subseteq \cdots$,按照前面的引理,
				
				一定有一个$k$,$\forall i \in \N,(a_k)=(a_{k+i})$,那么其实$a_k$就不可约,这样证明就完成了。
				
				现在证明它可以写成一系列不可约元的乘积。
				
				如果$a$不可约,那么命题成立;如果并非,有$a = p_1 c_1$,其中$p_1$不可约,我们如此不断拆分$c_i = p_{i+1}c_{i+1}$,如果有一步中$c_{n}$不可约,那么命题成立
				
				如果找不到这样的一步,那么按照整除的定义,有$(a) \subseteq (c_1) \subseteq \cdots \subseteq (c_k) \subseteq \cdots$,按照前面的引理,
				
				一定有一个$k$,$\forall i \in \N,(c_k)=(c_{k+i})$,那么其实$c_k$就不可约,这样证明就完成了。
			\end{proof}
			\begin{lemma}{}{}
				设$p \in R$是一个素元,$a \in R,a \neq 0$,那么$\exists n \in \N,p^n \mid a,p^{n+1}\nmid a$
			\end{lemma}
			\begin{proof}
				假设命题不成立,那么对于$\forall m \in N,p^m \mid a$。
				
				即,$\exists b_m,a = p^m b_m$
				
				因为$a = p^m b_m=p^{m+1} b_m$,所以$p b_{m+1} = b_m$,于是有$(b_0) \subseteq \cdots \subseteq (b_k) \subseteq \cdots$,
				
				按照引理,一定有一个$k$,$\forall i \in \N,(b_k)=(b_{k+i})$,这和$p$是素元的前提矛盾,于是命题得证。
			\end{proof}
				\begin{lemma}{}{}
				如果$p$为素元,$a,b\in R$,那么$ord_p ab=ord_p a + ord_p b$
			\end{lemma}
			\begin{proof}
				首先处理一些特殊情形:
				
				如果$a,b$中至少一者为$0$,不妨假设$a=0$,那么有$+\infty = +\infty + ord_p b$,因为$ord_p b$或者有限,或者为正无穷,于是命题成立;
				
				如果$p \nmid ab,p\nmid a,p\nmid b$,那么有$0=0+0$,命题成立;
				
				如果$p \mid ab$,但是$p \mid a$和$p \mid b$仅一者成立,那么不妨假设$p \nmid a$,于是依上面的引理,有$p \mid b$
				
				假设$ord_p b = n$,于是$\exists c,b=c\cdot p^n,p \nmid c$
				
				代入得:$ab=p^n\cdot (ac)$,但是$p \nmid a,p\nmid c$,那么$p \nmid ac$,即$ord_p ab=n$,于是$ord_p ab = 0+ ord_p b$,命题成立。
				
				最后假设$p \mid ab,p \mid a,p \mid b$,设$ord_p a = m,ord_p b = n$,于是$\exists c,d,a = c\cdot p^m,b=d\cdot p^n,p \nmid c,p\nmid d$
				
				那么$ab=(cd)\cdot p^{m+n}$,但是$p \nmid cd$,于是有$ord_p ab=ord_p a + ord_p b$,命题成立。
			\end{proof}
			最后,我们证明唯一分解定理
			\begin{them}{主理想整环的唯一分解定理}{}
				设$R$是一个主理想子环,那么$\forall n \in R,n \neq 0$
				\begin{equation}
					n = u \prod_{p} p^{ord_p n}
				\end{equation}
				其中$u$是$R$中的可逆元,求积符号对全体不可约元求积
			\end{them}
			\begin{proof}
				先对$n$不可逆考虑,此时有$n=\prod_{p} p^{a_p}$,其中$a_p$是未知的。
				
				取任意一个素元$q$,那么有$ord_q n=\sum\limits_{p} a_p\cdot ord_q (p)$
				
				只需注意到如果$q \neq p,ord_q (p)=0$,即得$a_q = ord_p (q)$,于是此情形下命题成立。
				
				对于$n$可逆的情形,$ord_p (n)=0$恒成立,于是命题也成立。
			\end{proof}
\ifx\allfiles\undefined
\end{document}
\fi