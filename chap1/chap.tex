\ifx\allfiles\undefined
\documentclass[12pt, a4paper, oneside, UTF8]{ctexbook}
\def\path{../config}
\usepackage{amsmath}
\usepackage{amsthm}
\usepackage{amssymb}
\usepackage{graphicx}
\usepackage{mathrsfs}
\usepackage{enumitem}
\usepackage{geometry}
\usepackage[colorlinks, linkcolor=black]{hyperref}
\usepackage{stackengine}
\usepackage{yhmath}
\usepackage{extarrows}
\usepackage{unicode-math}
\usepackage{tikz}
\usepackage{tikz-cd}
\usepackage{pifont}

\usepackage{fancyhdr}
\usepackage[dvipsnames, svgnames]{xcolor}
\usepackage{listings}

\definecolor{mygreen}{rgb}{0,0.6,0}
\definecolor{mygray}{rgb}{0.5,0.5,0.5}
\definecolor{mymauve}{rgb}{0.58,0,0.82}

\graphicspath{ {figure/},{../figure/}, {config/}, {../config/} }

\linespread{1.6}

\geometry{
    top=25.4mm, 
    bottom=25.4mm, 
    left=20mm, 
    right=20mm, 
    headheight=2.17cm, 
    headsep=4mm, 
    footskip=12mm
}

\setenumerate[1]{itemsep=5pt,partopsep=0pt,parsep=\parskip,topsep=5pt}
\setitemize[1]{itemsep=5pt,partopsep=0pt,parsep=\parskip,topsep=5pt}
\setdescription{itemsep=5pt,partopsep=0pt,parsep=\parskip,topsep=5pt}

\lstset{
    language=Mathematica,
    basicstyle=\tt,
    breaklines=true,
    keywordstyle=\bfseries\color{NavyBlue}, 
    emphstyle=\bfseries\color{Rhodamine},
    commentstyle=\itshape\color{black!50!white}, 
    stringstyle=\bfseries\color{PineGreen!90!black},
    columns=flexible,
    numbers=left,
    numberstyle=\footnotesize,
    frame=tb,
    breakatwhitespace=false,
} 
\usepackage[strict]{changepage} 
\usepackage{framed}
\usepackage{tcolorbox}
\tcbuselibrary{most}

\definecolor{greenshade}{rgb}{0.90,1,0.92}
\definecolor{redshade}{rgb}{1.00,0.88,0.88}
\definecolor{brownshade}{rgb}{0.99,0.95,0.9}
\definecolor{lilacshade}{rgb}{0.95,0.93,0.98}
\definecolor{orangeshade}{rgb}{1.00,0.88,0.82}
\definecolor{lightblueshade}{rgb}{0.8,0.92,1}
\definecolor{purple}{rgb}{0.81,0.85,1}

% #### 将 config.tex 中的定理环境的对应部分替换为如下内容
% 定义单独编号,其他四个共用一个编号计数 这里只列举了五种,其他可类似定义(未定义的使用原来的也可)
\newtcbtheorem[number within=section]{defn}%
{定义}{colback=OliveGreen!10,colframe=Green!70,fonttitle=\bfseries}{def}

\newtcbtheorem[number within=section]{lemma}%
{引理}{colback=Salmon!20,colframe=Salmon!90!Black,fonttitle=\bfseries}{lem}

% 使用另一个计数器 use counter from=lemma
\newtcbtheorem[use counter from=lemma, number within=section]{them}%
{定理}{colback=SeaGreen!10!CornflowerBlue!10,colframe=RoyalPurple!55!Aquamarine!100!,fonttitle=\bfseries}{them}

\newtcbtheorem[use counter from=lemma, number within=section]{criterion}%
{准则}{colback=green!5,colframe=green!35!black,fonttitle=\bfseries}{cri}

\newtcbtheorem[use counter from=lemma, number within=section]{corollary}%
{推论}{colback=Emerald!10,colframe=cyan!40!black,fonttitle=\bfseries}{cor}
% colback=red!5,colframe=red!75!black

% 这个颜色我不喜欢
%\newtcbtheorem[number within=section]{proposition}%
%{命题}{colback=red!5,colframe=red!75!black,fonttitle=\bfseries}{cor}

% .... 命题 例 注 证明 解 使用之前的就可以(全文都是这种框框就很丑了),也可以按照上述定义 ...
\renewenvironment{proof}{\par\textbf{证明:}\;}{\qed\par}
\newenvironment{solution}{\par{\textbf{解:}}\;}{\qed\par}
\newtheorem{proposition}{\indent 命题}[section]
\newtheorem{example}{\indent \color{SeaGreen}{例}}[section] % 绿色文字的 例 ,不需要就去除\color{SeaGreen}{}
\newtheorem*{rmk}{\indent 注}
\usepackage{amssymb}
\setmathfont{LatinModernMath-Regular}
\setmathfont[range=\mathbb]{TeXGyrePagellaMath-Regular}
\def\d{\mathrm{d}}
\def\R{\mathbb{R}}
\def\C{\mathbb{C}}
\def\Q{\mathbb{Q}}
\def\N{\mathbb{N}}
\def\Z{\mathbb{Z}}
\newcommand{\bs}[1]{\boldsymbol{#1}}
\newcommand{\ora}[1]{\overrightarrow{#1}}
\newcommand{\myspace}[1]{\par\vspace{#1\baselineskip}}
\newcommand{\xrowht}[2][0]{\addstackgap[.5\dimexpr#2\relax]{\vphantom{#1}}}
\newenvironment{ca}[1][1]{\linespread{#1} \selectfont \begin{cases}}{\end{cases}}
\newenvironment{vx}[1][1]{\linespread{#1} \selectfont \begin{vmatrix}}{\end{vmatrix}}
\newcommand{\tabincell}[2]{\begin{tabular}{@{}#1@{}}#2\end{tabular}}
\newcommand{\pll}{\kern 0.56em/\kern -0.8em /\kern 0.56em}
\newcommand{\dive}[1][F]{\mathrm{div}\;\bs{#1}}
\newcommand{\rotn}[1][A]{\mathrm{rot}\;\bs{#1}}
\usepackage{xeCJK}
\setCJKmainfont{SimSun}[BoldFont={SimHei}, ItalicFont={KaiTi}] % 设置中文支持

\newcommand{\point}[1]{\item {#1}}
\newenvironment{para}[1]{%
\ifcase#1\relax
\begin{enumerate}[label=\arabic*.] % 1.2.3.
\or
\begin{enumerate}[label=\textcircled{\arabic*}] % ①②③
\or
\begin{enumerate}[label=(\roman*)] % (i)(ii)(iii)
\else
\begin{enumerate}[label=\arabic*.] % 默认格式
\fi
}{
\end{enumerate}
}

\def\myIndex{0}
% \input{\path/cover_package_\myIndex.tex}

\def\myTitle{初等数论笔记}
\def\myAuthor{Zhang Liang}
\def\myDateCover{\today}
\def\myDateForeword{\today}
\def\myForeword{前言标题}
\def\myForewordText{
    前言内容
}
\def\mySubheading{副标题}


\begin{document}
% \input{\path/cover_text_\myIndex.tex}

\newpage
\thispagestyle{empty}
\begin{center}
    \Huge\textbf{\myForeword}
\end{center}
\myForewordText
\begin{flushright}
    \begin{tabular}{c}
        \myDateForeword
    \end{tabular}
\end{flushright}

\newpage
\pagestyle{plain}
\setcounter{page}{1}
\pagenumbering{Roman}
\tableofcontents

\newpage
\pagenumbering{arabic}
\setcounter{chapter}{0}
\setcounter{page}{0}

\pagestyle{fancy}
\fancyfoot[C]{\thepage}
\renewcommand{\headrulewidth}{0.4pt}
\renewcommand{\footrulewidth}{0pt}








\else
\fi
%标题
\chapter{唯一分解}
	\section{$\Z$上的唯一分解}
		\subsection{整除和素数}
			作为数论的基础,我们需要先研究素数。
			\begin{defn}{整除}{}
				设$a,b \in \Z,\exists c \in \Z,\text{s.t. }ac=b$,那么我们称$a$整除$b$,记作$a \mid b$;
				
				否则,我们称$a$不整除$b$,记作$a \nmid b$
			\end{defn}
			有了整除的定义,我们就可以定义素数的概念:
			\begin{defn}{素数}{}
				设$p \in \N,p \geqslant 2$,如果$a \mid p,a \in \N \Leftrightarrow a = 1 \vee a = p$
				
				那么我们称$p$是一个素数,否则称其是个合数。
			\end{defn}
			我们研究素数的第一个目的是一个比较显然的事实:所有大于1的整数都能写成若干素数的乘积。我们还将看到,这个分解是唯一的。
			
			我们先引入以下概念。
			\begin{defn}{素数的指数}{}
				设有素数$p \in \N$和$n \in \Z$
				
				如果$p \mid n$,那么我们记满足$p^k \mid n$的最大正整数$k$为$ord_p n$,称为$p$的指数
				
				特别地,如果$p \nmid n$,我们定义$ord_p n = 0$;如果$n =0$,我们定义$ord_p n = +\infty$
			\end{defn}
		\subsection{算数基本定理}
			我们先给出一个基本引理,它证明了可分解但是没有证明唯一性。
			\begin{lemma}{大于1的正整数的可分解性}{}
				每一个大于$1$的正整数都可以写成有限个素数的乘积。
			\end{lemma}
			\begin{proof}
				假设命题不成立。于是一定有一个$N$为不满足此性质的最小正整数。
				
				首先,$N$不可能是素数,否则它就可以写成自身,与假设矛盾。
				
				于是,一定$\exists m,n,1<m < N,1<n<N$,满足$mn=N$
				
				那么,$m,n$也不能满足命题所说的性质,否则$N$也就有了这一性质,与假设矛盾。
				
				但是,这样$N$就不再是最小的满足此性质的正整数了。与假设矛盾,命题得证
			\end{proof}
			接下来我们证明唯一性。为了证明这一部分,我们将提出一系列全新概念,包括互素、最大公因数等。
			\begin{defn}{$\Z$上生成的子模}{}
				我们定义:
				
				$\Z \text{-span} (a_1,\cdots,a_n) := \{a_1\cdot q_1 +\cdots+a_n\cdot q_n | q_i \in \Z\}$
				
				称为$\Z$上由$a_1,\cdots,a_n$生成的子模
			\end{defn}
			我们先证明一个引理:由两个数在整数集上生成的子模,其实也可以只由一个数生成。
			\begin{lemma}{}{}
				对于$\Z\text{-span}(a,b),\exists c \in \Z,\text{s.t. }\Z\text{-span}(a,b)=\Z\text{-span}(c)$
			\end{lemma}
			\begin{proof}
				如果$a=b=0$,那么命题是平凡的。
				
				如果$a,b$中至少有一者不为$0$,那么一定能找到$\Z\text{-span}(a,b)$的最小正元素$c$。
				
				由定义易知,$\Z\text{-span}(c) \subseteq \Z\text{-span}(a,b)$
				
				现在取$\forall d \in \Z\text{-span}(a,b)$,那么一定$\exists q,r \in \Z,0 \leqslant r \leqslant c,d=qc+r$
				
				那么$r=d-qc$,因为$c,d\in \Z\text{-span}(a,b)$,于是$r \in \Z\text{-span}(a,b)$
				
				于是一定有$r=0$因为我们已经假设$c$是$\Z\text{-span}(a,b)$最小的正整数,而$0 \leqslant r$。
				
				那么$d=qc \in \Z\text{-span}(a,b)$,于是$\Z\text{-span}(a,b) \subseteq \Z\text{-span}(c)$,于是命题得证。
			\end{proof}
			这个数实际上就是两个数的最大公因数,我们先给出定义,随后给出证明
			\begin{defn}{公因数}
				对于$a,b \in \Z$,如果$c\in\Z$满足$c\mid a,c\mid b$,那么称$c$是$a,b$的最大公因数。
				
				如果又有:对于任何$a,b$的公因数$d,d \mid c$,那么称$c$是$a,b$的最大公因数,记作$(a,b)$
			\end{defn}
			\begin{lemma}{}{}
				如果$\Z\text{-span}(a,b)=\Z\text{-span}(c)$,那么$c$是$a,b$的最大公因数
			\end{lemma}
			\begin{proof}
				取$a,b$的一个公因数$d\mid a,d\mid b$,由$\Z\text{-span}(a,b)$的定义可知,$\forall e \in \Z\text{-span}(a,b),d \mid e$
				
				那么$\forall e \in \Z\text{-span}(c),d \mid e$,那么$d \mid c$,命题得证。
			\end{proof}
			最大公因数是$1$的情况比较特别,我们称之为互素
			\begin{defn}{互素}{}
				如果$a,b$的公因数仅有$1,-1$,那么我们称$a,b$互素,记作$(a,b)=1$
			\end{defn}
			互素有以下显然的性质:
			\begin{lemma}{}{}
				$(a,b)=1 \Leftrightarrow \exists r,q \in \Z,\text{s.t. }ra+qb=1$
			\end{lemma}
			\begin{proof}
				只需注意到$(a,b)=1 \Leftrightarrow \Z\text{-span}(a,b)=\Z\text{-span}(1)$,于是命题得证。
			\end{proof}
			\begin{lemma}{}{}
				若$a \mid bc,(a,b)=1$,那么$a\mid c$
			\end{lemma}
			\begin{proof}
				因为$(a,b)=1$,所以$\exists r,q\in \Z,\text{s,t. }ra+qb=1$
				
				所以$rac+qbc=c$,但是我们知道$a\mid rac,a\mid qbc$,于是$a\mid c$
			\end{proof}
			\begin{lemma}{}{}
				如果$p$是一个素数,$p \mid bc$,那么$p \mid b,p\mid c$至少有一者成立。
			\end{lemma}
			\begin{proof}
				因为$p$是一个素数,所以$(p,b)$或者为$1$,或者为$p$。
				
				如果$(p,b)=p$那么一定有$p \mid b$,命题成立;
				
				如果$(p,b)=1$,那么$p,b$的公因数仅有$1,-1$,那么必定有$p \mid c$
			\end{proof}
			最后,为了指出唯一分解中素数的指数,我们给出最后一个引理。
			\begin{lemma}{}{}
				如果$p$为素数,$a,b\in \Z$,那么$ord_p ab=ord_p a + ord_p b$
			\end{lemma}
			\begin{proof}
				首先处理一些特殊情形:
				
				如果$a,b$中至少一者为$0$,不妨假设$a=0$,那么有$+\infty = +\infty + ord_p b$,因为$ord_p b$或者有限,或者为正无穷,于是命题成立;
				
				如果$p \nmid ab,p\nmid a,p\nmid b$,那么有$0=0+0$,命题成立;
				
				如果$p \mid ab$,但是$p \mid a$和$p \mid b$仅一者成立,那么不妨假设$p \nmid a$,于是依上面的引理,有$p \mid b$
				
				假设$ord_p b = n$,于是$\exists c,b=c\cdot p^n,p \nmid c$
				
				代入得:$ab=p^n\cdot (ac)$,但是$p \nmid a,p\nmid c$,那么$p \nmid ac$,即$ord_p ab=n$,于是$ord_p ab = 0+ ord_p b$,命题成立。
				
				最后假设$p \mid ab,p \mid a,p \mid b$,设$ord_p a = m,ord_p b = n$,于是$\exists c,d,a = c\cdot p^m,b=d\cdot p^n,p \nmid c,p\nmid d$
				
				那么$ab=(cd)\cdot p^{m+n}$,但是$p \nmid cd$,于是有$ord_p ab=ord_p a + ord_p b$,命题成立。
			\end{proof}
			至此,我们可以开始证明算数基本定理了。
			\begin{them}{算数基本定理}{}
				$\forall n \in \Z,n \neq 0$
				
				\begin{equation}
					n = (\text{sgn } n) \prod_{p} p^{ord_p |n|}
				\end{equation}
				其中求积下标$p$对全体素数求积
			\end{them}
			\begin{proof}
				先对$n > 1$考虑,此时有$n=\prod_{p} p^{a_p}$,其中$a_p$是未知的。
				
				取任意一个素数$q$,那么有$ord_q n=\sum\limits_{p} a_p\cdot ord_q (p)$
				
				只需注意到如果$q \neq p,ord_q (p)=0$,即得$a_q = ord_p (q)$,于是此情形下命题成立。
				
				对于$n=1$的情形,$ord_p (1)=0$恒成立,于是命题也成立。
				
				$n <0$的情形,在已经证明了上述事实后是显然的。 
			\end{proof}
\ifx\allfiles\undefined
\end{document}
\fi